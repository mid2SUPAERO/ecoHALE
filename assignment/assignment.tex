\documentclass{article}
\usepackage{fullpage}
\usepackage{hyperref}

\title{MDO assignment; v. 1.1}
%\subtitle{Aerostructural optimization}
\author{John T. Hwang, Justin S. Gray, John P. Jasa, and Joaquim R. R. A. Martins}

\newcommand\be{\begin{enumerate}}
\newcommand\ee{\end{enumerate}}
\newcommand\code[1]{\texttt{#1}}

\begin{document}

	\maketitle

	For all problems in this assignment, we will use the \code{problems.py} file.
	By supplying command-line arguments to the file, you can choose which
	problem to run.
	For example, \code{python problems.py prob1} runs the code for problem 1.
	Available options are \code{prob1, prob2, prob3ab, prob3c}.

	\be

		\item \textbf{Structural optimization.}
		\be
			\setlength\itemsep{1em}
			\item This script performs structural analysis and optimization of
			a tubular beam clamped in the middle.
			Run the optimization, first with uniform loading
			and then again with tip loads applied.
			You must change the code where we set the \code{loads} variable
			to use tip loads.
			What optimized thickness distributions do you see for each case?\\
			Commands:
			\be
				\item run the optimization: \code{python problems.py prob1}
				\item view the results: \code{python plot\_all.py s}
				\item view the optimization history: \code{python OptView.py s}

			\ee

			\item Run the optimization with tip loads applied for
			a range of different mesh sizes by changing the \code{num\_y} value.
			Plot the computation time vs \code{num\_y}.
			\item The script produces an html file, \code{prob1.html},
			that can be useful for studying the problem structure. You can
			open this file in any web browser.
			What is the physical interpretation of this problem?
			That is, what are we minimizing and subject to what constraint?
		\ee

		\item \textbf{Multidisciplinary analysis.} Couple aerodynamics and structures together.
		\be
			\setlength\itemsep{1em}
			\item Open \code{aerostruct.html} to use a guide.
			Assemble the aerostructural analysis group following the layout presented
			there. For this problem run with \code{python problems.py prob2}.
			\item Now that you have assembled the analysis groups correctly in part (a),
			we will try different nonlinear solvers.
			Look at \href{http://openmdao.readthedocs.io/en/latest/srcdocs/packages/openmdao.solvers.html}
			{this OpenMDAO documentation page}
			to see what solvers are available and how to use them.
			Note that we are focusing on nonlinear solvers here and will examine linear
			solvers in part (c).

			Try NLGS, Newton, and hybrid NLGS/Newton for
			\code{num\_y = 9} and \code{num\_y = 13} then compare run times with these different solvers.
			Then try to run the problem with the Newton solver in the \code{root} group
			instead of the \code{coupled} group.
			Why do we normally put the nonlinear solver on the \code{coupled} group
			instead of the \code{root} group?

			\item Again visit the OpenMDAO documentation page to see what linear solvers
			are available.
			Try LNGS, Krylov, Krylov-PC-GS, and direct linear solvers while
			using the Newton nonlinear solver.
			Which ones can successfully converge the linear problem?
			Which one gives the fastest convergence for the Newton solver?
			Why should we not we use the \code{DirectSolver} with high-fidelity problems?
		\ee

		\item \textbf{Multidisciplinary optimization.} Now that you've set up
		the aerostructural analysis, you're ready to try aerostructural	optimization.
		\be
			\setlength\itemsep{1em}
			\item Compute the analytic derivatives of the multidisciplinary system
			by running \code{python problems.py prob3ab}.
			Take note of the run times and derivatives values output by the run script.
			You can visualize the system by running \code{python plot\_all.py as}.

			\item Compute the same derivatives using finite differences
			and compare the timings for different \code{num\_y} values.
			Use the same code, but add a few lines to force the system to use
			finite-differencing to compute the derivatives.
			Examine \href{http://openmdao.readthedocs.io/en/latest/usr-guide/examples/example_fd.html}
			{this OpenMDAO documentation page} to see how to use finite-differencing.

			\item We will now perform aerostructural optimization.
			Add the following design variables in the appropriate locations:
			\begin{itemize}
				\item `twist', lower = -10, upper = 10
				\item `alpha', lower = -10, upper = 10
				\item `thickness', lower = 0.003, upper = 0.025, scaler = 1000
			\end{itemize}
			and the follow objective and two constraints respectively:
			\begin{itemize}
				\item `fuelburn'
				\item `failure', upper = 0
				\item `eq\_con', equals = 0
			\end{itemize}
			Now run the code using \code{python problems.py prob3c}.
			This optimization will take some time, but you can monitor the progress while it runs.
			Without stopping the optimization, open a second command window and type the command:

			\code{python OptView.py as}.

			You can change the settings to adjust what variables you're plotting and you can check the
			\code{Automatically refresh} option to have OptView update the plots
			as new iterations are saved.

			You can also open a 3D visualization of your wing by typing the command:

			\code{python plot\_all.py as}

			\item Now that you've run the aerostructural optimization case,
			experiment with different solver parameters and design variable options
			to find the optimum in the fewest number of function evaluations.
			You must keep the same design variables, objective, and constraints as in part (c), but
			you are free to vary the solver setup, parameter scaling, and optimizer settings.
			Refer to the previous OpenMDAO documentation pages for more information about possible options.

			Your final fuelburn measure must be below 94477 kg for the optimization to be considered
			successful.
			The top five groups who use the fewest function evaluations will receive extra credit.
			10 points for number 1, 8 points for number 2, and so on.

		\ee


	\ee

\end{document}
